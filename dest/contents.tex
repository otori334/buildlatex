\documentclass[uplatex,dvipdfmx,a4paper,10pt]{jsarticle}
\usepackage{geometry}
%\geometry{left=3cm, right=2cm, top=2cm, bottom=3cm}

%\usepackage[draft]{graphicx, color}
%https://qiita.com/zr_tex8r/items/442b75b452b11bee8049
%\usepackage\usepackage{graphicx}

% dvipdfmx は自動的に有効になる{graphicx}% dvipdfmx は自動的に有効になる
\usepackage{graphicx}

% dvipdfmx は自動的に有効になる% dvipdfmx は自動的に有効になる
\usepackage[dvipdfmx]{color}% dvipdfmx は自動的に有効になるらしい

\usepackage{multicol}
\usepackage{tikz}
\usepackage{float}
\usepackage[version=3]{mhchem}
\usepackage{chemfig}
\usepackage{enumerate}
\usepackage{bigfoot}
\usepackage{longtable} % 表組みに必要
\usepackage{booktabs} % 表組みに必要
\usepackage{subfig} % 図の横並び表示に必要
\usepackage{siunitx} % SI単位(国際単位系)を出力
\usepackage{here} 

\usepackage[%
	hidelinks,%
	pdfusetitle,%
	colorlinks=false,%
	bookmarks=true, % 以下ブックマークに関する設定
	bookmarksnumbered=true,%
	pdfborder={0 0 0},%
	bookmarkstype=toc,%
	pdftitle={「反応速度論」レポート課題 No.~1},%
	pdfauthor={otori334}%
]{hyperref}
\title{ 「反応速度論」レポート課題 No.~1}
\author{33417334 otori 334}
\usepackage{pxjahyper}
\usepackage{url}
\urlstyle{sf}


\usepackage{comment}

%\usepackage[backend=bibtex]{biblatex}
%\usepackage[backend=bibtex,style=numeric]{biblatex}
%\usepackage[backend=bibtex,style=ieee]{biblatex}
\usepackage[backend=biber,style=ieee]{biblatex}
%\bibliography{sample.bib}
\DeclarePrefChars{'-} % https://oku.edu.mie-u.ac.jp/tex/mod/forum/discuss.php?d=2313
\addbibresource{references.bib}



\newcommand*{\ppa}{%
	\ifthenelse{\boolean{mmode}}%
		{\mathrm{p}K_\mathrm{a}}%
		{\(\mathrm{p}K_\mathrm{a}\)}%
	}%

\newcommand*{\ppb}{%
	\ifthenelse{\boolean{mmode}}%
		{\mathrm{p}K_\mathrm{b}}%
		{\(\mathrm{p}K_\mathrm{b}\)}%
	}%

\newcommand*{\pph}{%
	\ifthenelse{\boolean{mmode}}%
		{\mathrm{pH}}%
		{\(\mathrm{pH}\)}%
	}%

\newcommand*{\diff}{%
	\ifthenelse{\boolean{mmode}}%
		{\mathrm{d}}%
		{\(\mathrm{d}\)}%
	}%

\usepackage{mathrsfs}


%https://orumin.blogspot.com/2017/09/biblatex.html
\AtEveryBibitem{%
\ifentrytype{article}{%
}{}
\ifentrytype{inproceedings}{%
\clearfield{volume}
\clearfield{number}
}{}
\clearlist{publisher}
\clearfield{isbn}
\clearlist{location}
\clearfield{doi}
\clearfield{url}
}

\renewbibmacro{in:}{%
\ifentrytype{inproceedings}{%
  \setunit{}
  \addperiod\addspace In \textit{Proc.\ of the}}%
{\printtext{\bibstring{in}\intitlepunct}}
}

\renewbibmacro*{series+number:emphcond}{%
\ifentrytype{inproceedings}{%
  \printtext{(}\printfield{series}\printtext{)}\setunit*{\addcomma\space}\newunit}%
{%
  \printfield{number}%
  \setunit*{\addspace\bibstring{inseries}\addspace}%
  \ifboolexpr{%
    not test {\iffieldundef{volume}}
  }%
   {\printfield{series}}%
   {\ifboolexpr{%
       test {\iffieldundef{volume}}
       and
       test {\iffieldundef{part}}
       and
       test {\iffieldundef{number}}
       and
       test {\ifentrytype{book}}
     }%
      {\newunit\newblock}%
      {}%
    \printfield[noformat]{series}}%
  \newunit}
}




\usepackage[jis2004]{otf}


% prevent hyphenation
\hyphenpenalty=10000\relax
\exhyphenpenalty=10000\relax
\sloppy


%行間調整
%\renewcommand\baselinestretch{0.8} 

\usepackage{setspace} %% for spacing
%\begin{spacing}{0.7}
%hogehogehogehoge
%\end{spacing}

%脚注番号を (1)に変更
%\renewcommand\thefootnote{(\arabic{footnote})}
%脚注番号を 1)に変更
\renewcommand\thefootnote{\arabic{footnote})}
\newcommand{\vect}[1]{\mbox{\boldmath \(#1\)}}
\newcommand*{\uni}[2][]{
	\ifthenelse{\boolean{mmode}}{#1\,\si{#2}}{\(#1\,\si{#2}\)}}
	
\renewcommand{\abstractname}{}
	
	
%\newcommand*{\myfig}[4][width=10cm]{\begin{figure}[htb][H]\centering %\includegraphics[width=1.0\columnwidth][#1]{#2} \caption{#3} \label{fig:#4} \end{figure}}


% http://nos.hateblo.jp/entry/20081015/1224084491
%\newcommand{\fig}[3][width=15cm]{
%\begin{figure}[htb][ht]
%\begin{center}
% \includegraphics[width=1.0\columnwidth][#1]{figs/#2}
%\end{center}
%\caption{#3}
%\label{fig:#2}
%\end{figure}}

%\newcommand{\fref}[1]{図\ref{fig:#1}}

%\renewcommand{\abstractname}{概要}

\def\tightlist{\itemsep1pt\parskip0pt\parsep0pt}
%\date{\date{提出日:\today}}
\date{提出日:\today}
%\setatomsep{1cm}
%\usepackage{luatexja-ruby}
% https://qiita.com/Selene-Misso/items/6c27a4a0947f10af3119o
%\subtitle{\vspace{10truept}{\large }}
%%%%%%%%%%%%%%%%%%%%%%%%%%%%%%%%%%%%%%%%%%%%%%%%%%%%%%%%%

\begin{document}

\maketitle

\begin{abstract}\end{abstract}
	


\thispagestyle{empty}



%\input{automatic_generated}
\begin{enumerate}
\def\labelenumi{\arabic{enumi}.}
\item
  完全気体を仮定して,\SI{300}{\kelvin}における単原子分子の平均の並進運動エネルギー\(\bigl(\frac{1}{2}mc^2\bigr)\)を求めよ.ただし,\(m\)は気体分子1個の質量,\(c^2\)は平均2乗速さである.

  並進運動エネルギーは\(\frac{3}{2}kT\)で表せるから \begin{eqnarray*}
\frac{1}{2}mc^2 &=& \frac{3}{2}kT \\
&=& \frac{3}{2} \cdot \uni[1.38 \times 10^{-23}]{J.K^{-1}} \cdot \uni[300]{K} \\
&=& \uni[6.21\times 10^{-21}]{J}
\label{}
\end{eqnarray*}
\item
  速さに関するMaxwell分布関数\(f(v)\)を用いて,以下の問に答えよ.

  Maxwell分布関数は \begin{eqnarray*}
f(v) = 4\pi {\Bigl(\frac{M}{2\pi RT}\Bigr) }^\frac{3}{2} v^{2} \exp \Bigl( - \frac{Mv^{2}}{2RT} \Bigr)
\label{}
\end{eqnarray*}

  \begin{enumerate}
  \def\labelenumii{\arabic{enumii}.}
  \tightlist
  \item
    \(f(v)\)を用いて,根平均2乗速さ\(c\),平均の速さ\(\overline{c}\),最確の速さ\(c^{\ast}\)を与える式を導け.
  \end{enumerate}

  最確の速さ\(c^{\ast}\)はMaxwell分布関数の極大値で示されるから条件は
  \begin{eqnarray*}
  \frac{\diff f(v)}{\diff v} = 0
\label{}
\end{eqnarray*}

  計算すると \begin{eqnarray*}
f'(v) &=& 4\pi {
    \Bigl(
        \frac{M}{2\pi RT}
    \Bigr) 
}^\frac{3}{2} 
    \Bigl\{ 
        2v \exp 
        \Bigl( 
          - \frac{Mv^{2}}{2RT} 
        \Bigr) + v^2 
        \Bigl( 
          - \frac{Mv}{2RT} 
        \Bigr) 
        \exp 
        \Bigl( 
          - \frac{Mv^{2}}{2RT} 
        \Bigr) 
    \Bigr\} \\
&=& 4\pi {
    \Bigl(
        \frac{M}{2\pi RT}
    \Bigr) 
}^\frac{3}{2} 
    \Bigl( 
        2v - \frac{Mv^{3}}{2RT}
    \Bigr)
\exp 
    \Bigl( 
        - \frac{Mv^{2}}{2RT} 
    \Bigr) 
\label{}
\end{eqnarray*}

  条件より \begin{eqnarray*}
0 &=& 
    \Bigl( 
        2c^{\ast} - \frac{Mc^{\ast 3}}{2RT} 
    \Bigr) \\
c^{\ast} &=& 
    {
        \Bigl( 
            \frac{2RT}{M}
        \Bigr) 
    }^{\frac{1}{2}}
\label{}
\end{eqnarray*}

  平均の速さ\(\overline{c}\)は \begin{eqnarray*}
\overline{c} &=& \int_{0}^{\infty} 
vf(v)
\,\diff v \\
&=& 
4\pi {
    \Bigl(
        \frac{M}{2\pi RT}
    \Bigr) 
}^\frac{3}{2} 
\int_{0}^{\infty} 
    v^{3} 
    \exp 
    \Bigl( 
      - \frac{Mv^{2}}{2RT} 
    \Bigr) 
\,\diff v 
\\ &=& 
    4\pi {
        \Bigl(
            \frac{M}{2\pi RT}
        \Bigr) 
    }^\frac{3}{2} 
\cdot
\frac{1}{2}
    {
        \Bigl(
            \frac{M}{2 RT}
        \Bigr) 
    }^{-2} 
\\ &=&
\frac{2}{\sqrt{\pi}}
\cdot
{
    \Bigl(
        \frac{M}{2 RT}
    \Bigr) 
}^{- \frac{1}{2} }
\\ &=&
{
    \Bigl(
        \frac{8 RT}{\pi M}
    \Bigr) 
}^{\frac{1}{2} }
\label{}
\end{eqnarray*}

  平均2乗速さ\(c^2\)は

  \begin{eqnarray*}
c^2
&=&
\int_{0}^{\infty} 
    v^2 f(v)
\,\diff v 
\\ &=&
4\pi {
    \Bigl(
        \frac{M}{2\pi RT}
    \Bigr) 
}^\frac{3}{2} 
\int_{0}^{\infty} 
    v^4 
\exp 
\Bigl( 
  - \frac{Mv^{2}}{2RT} 
\Bigr) 
\,\diff v
\\ &=&
4\pi {
  \Bigl(
    \frac{M}{2\pi RT}
    \Bigr) 
}^\frac{3}{2} 
\cdot
\frac{3}{8}
    {
        \Bigl(
            \frac{M}{2 RT}
        \Bigr) 
    }^{-\frac{5}{2}} 
\\ &=&
\frac{3}{2}
{
    \Bigl(
        \frac{M}{2 RT}
    \Bigr) 
}^{-1} 
\label{}
\end{eqnarray*}

  よって根平均2乗速さ\(c\)は \begin{eqnarray*}
c =
{
    \Bigl(
        \frac{3 RT}{M}
    \Bigr) 
}^{\frac{1}{2}} 
\label{}
\end{eqnarray*}


  \begin{enumerate}
  \def\labelenumii{\arabic{enumii}.}
  \setcounter{enumii}{1}
  \tightlist
  \item
    \SI{298}{\kelvin}における\ce{He}分子(モル質量\SI{4.0}{g.mol^{-1}})の根平均2乗速さ\(c\),平均の速さ\(\overline{c}\),最確の速さ\(c^{\ast}\)を求めよ.
  \end{enumerate}

  \begin{eqnarray*}
c &=&
{
\Bigl(
\frac{3 RT}{M}
\Bigr) 
}^\frac{1}{2}
\\ &=&
{
\Bigl(
\frac{3 \cdot \uni[8.31]{J.K^{-1}.mol^{-1}}  \cdot
\uni[298]{K}}{\uni[4.0 \times 10^{-3}]{kg.mol^{-1}}}
\Bigr) 
}^\frac{1}{2} 
\\ &\approx &
\uni[1362.8]{m.s^{-1}}
\\
\overline{c} &=&
{
\Bigl(
\frac{8 RT}{\pi M}
\Bigr) 
}^\frac{1}{2}
\\ &=&
{
\Bigl(
\frac{8 \cdot \uni[8.31]{J.K^{-1}.mol^{-1}}  \cdot
\uni[298]{K}}{\pi \cdot \uni[4.0 \times 10^{-3}]{kg.mol^{-1}}}
\Bigr) 
}^\frac{1}{2} 
\\ &\approx &
\uni[1256]{m.s^{-1}}
\\
c^{\ast} &=&
{
\Bigl(
\frac{2 RT}{M}
\Bigr) 
}^\frac{1}{2}
\\ &=&
{
\Bigl(
\frac{2 \cdot \uni[8.31]{J.K^{-1}.mol^{-1}}  \cdot
\uni[298]{K}}{\uni[4.0 \times 10^{-3}]{kg.mol^{-1}}}
\Bigr) 
}^\frac{1}{2} 
\\ &\approx &
\uni[1113]{m.s^{-1}}
\label{}
\end{eqnarray*}

  \begin{enumerate}
  \def\labelenumii{\arabic{enumii}.}
  \setcounter{enumii}{2}
  \tightlist
  \item
    \SI{298}{\kelvin}における\ce{He}分子が根平均2乗速さ\(c\)から\(\pm\SI{4.0}{m.s^{-1}}\)の範囲にある確率はいくらか.その速度範囲で\(f(v)\)が一定であると近似して求めよ.
  \end{enumerate}

  \begin{eqnarray*}
f(v)\diff v 
&=&
4\pi 
{
\Bigl(
\frac
{\uni[4.0 \times 10^{3}]{kg.mol^{-1}}}
{2\pi \cdot \uni[8.31]{J.K^{-1}.mol^{-1}}  \cdot \uni[298]{K}}
\Bigr) 
}^\frac{3}{2} 
\cdot
c^2
\cdot
\exp 
\Bigl( 
- \frac
{\uni[4.0 \times 10^{3}]{kg.mol^{-1}} \cdot c^2}
{2 \cdot \uni[8.31]{J.K^{-1}.mol^{-1}}  \cdot \uni[298]{K}}
\Bigr) 
\cdot
\bigl\{
    10 - (-10)
\bigr\}
\\ &=&
\uni[1.36 \times 10^{-2}]{}
\\ &=&
1.36\si{\percent}
\\ &\approx&
1\si{\percent}
\label{}
\end{eqnarray*}

\end{enumerate}

\begin{enumerate}
\def\labelenumi{\arabic{enumi}.}
\setcounter{enumi}{2}
\item
  \(1\,\mathrm{atm}\),\SI{298}{\kelvin}において,1個の\ce{N2}分子(分子直径\SI{0.37}{nm},モル質量\SI{28}{g.mol^{-1}})の衝突頻度\(z\)および衝突密度\(Z_\mathrm{AA}\)はいくらか.また,平均自由行程\(\lambda\)の値も計算せよ.ただし,\(\SI{1}{atm} = 1.01 \times 10^5 \,\si{\pascal}= 1.01 \times 10^5 \si{J.m^{-3}}\)
  である.

  衝突頻度\(z\)は

  \begin{eqnarray*}
z &=& 
\sqrt{2} \, \overline{c} \cdot \pi d^2 \frac{NP}{KT}
\\ &=& 
\sqrt{2} \cdot 
{
\Bigl(
\frac
{
8 \cdot \uni[8.31]{J.K^{-1}.mol^{-1}} \cdot \uni[298]{K}
}
{
\pi \cdot \uni[28 \times 10^{-3}]{kg.mol^{-1}}
}
\Bigr)
}
^{\frac{1}{2}}
\cdot 
\pi
\cdot 
(
\uni[0.37 \times 10^{-9}]{m}
)^{2}
\cdot 
\frac
{1 \cdot \uni[1.01 \times 10^5]{J.m^{-3}} }{\uni[1.38 \times 10^{-23}]{J.K^{-1}} \cdot \uni[298]{K}}
\\ &=&
\uni[7.09 \times 10^9]{s^{-1}}
\\ &\approx&
\uni[7.1 \times 10^9]{s^{-1}}
\label{}
\end{eqnarray*}



  衝突密度\(Z_\mathrm{AA}\)は

  \begin{eqnarray*}
Z_\mathrm{AA}
&=&
\frac
{
\sqrt{2} \, \overline{c} 
}
{2}
\cdot \pi d^2 
{
\Bigl(
\frac{P}{KT}
\Bigr)
}^{2}
\\ &=& 
{
\Bigl(
\frac
{
8 \cdot \uni[8.31]{J.K^{-1}.mol^{-1}} \cdot \uni[298]{K}
}
{
2 \pi \cdot \uni[28 \times 10^{-3}]{kg.mol^{-1}}
}
\Bigr)
}
^{\frac{1}{2}}
\cdot 
\pi
\cdot 
(
\uni[0.37 \times 10^{-9}]{m}
)^{2}
\cdot
{
\Bigl( 
\frac
{1 \cdot \uni[1.01 \times 10^5]{J.m^{-3}} }
{\uni[1.38 \times 10^{-23}]{J.K^{-1}} \cdot \uni[298]{K}}
\Bigr)
}^2
\\ &=&
\uni[8.71 \times 10^{34}]{s^{-1}.m^{3}}
\\ &\approx&
\uni[8.7 \times 10^{34}]{s^{-1}.m^{3}}
\label{}
\end{eqnarray*}


  平均自由行程\(\lambda\)は

  \begin{eqnarray*}
\lambda
&=&
\frac
{
\overline{c} 
}
{
z
}
=
\frac{
KT
}{
\sqrt{2}\rho P
}
\\ &=&
\frac{
\uni[1.38 \times 10^{-23}]{J.K^{-1}}
\cdot 
\uni[298]{K}
}{
\sqrt{2} \pi 
\cdot 
(
\uni[0.37 \times 10^{-9}]{m}
)^{2}
\cdot
\uni[1.01 \times 10^5]{J.m^{-3}}
}
\\ &=&
\uni[6.69 \times 10^{-8}]{m}
\\ &\approx&
\uni[6.7 \times 10^{-8}]{m}
\label{}
\end{eqnarray*}

\item
  \SI{300}{\kelvin}において,\ce{N2}分子の平均自由行程が分子直径の10倍になるときの圧力(単位:\(\mathrm{atm}\))を求めよ.

  条件

  \begin{eqnarray*}
\lambda =\frac{KT}{\sqrt{2} \rho P} = 10d
\label{}
\end{eqnarray*}


  \(P\)は

  \begin{eqnarray*}
P 
&=&
\frac{KT}{10\sqrt{2} \pi d^3} 
\\ &=& 
\frac{\uni[1.38 \times 10^{-23}]{J.K^{-1}} \times \uni[300]{K}}{10\sqrt{2} \pi 
(
\uni[0.37 \times 10^{-9}]{m}
)^{3}
}
\\ &=&
\uni[1.84 \times 10^6]{J.m^{-3}}
\\ &=&
\uni[1.84 \times 10^6]{Pa}
\\ &=&
\uni[18.2]{atm}
\label{}
\end{eqnarray*}

\end{enumerate}

\begin{enumerate}
\def\labelenumi{\arabic{enumi}.}
\setcounter{enumi}{4}
\item
  \SI{300}{\kelvin}において,\SI{1.0}{atm}の空気中で\ce{O2}分子(分子直径\SI{0.357}{nm},モル質量\SI{32}{g.mol^{-1}})と\ce{N2}分子(分子直径\SI{0.37}{nm},モル質量\SI{28}{g.mol^{-1}})の衝突密度を計算せよ.

  \begin{eqnarray*}
Z_\mathrm{AB} &=& 
\pi d^2 \overline{c} 
\Bigl(
\frac{N_{\ce{O2}}}{V}
\Bigr)
\cdot
\Bigl(
\frac{N_{\ce{N2}}}{V}
\Bigr)
\\ &=&
\pi 
{
\Bigl(
\uni[
\frac{0.357 + 0.37}{2}
]{nm}
\Bigr)
}^2
\cdot
{
\Bigl\{
\frac
{
8 
\cdot \uni[(28 + 32) \times 10^{-3}]{kg.mol^{-1}}
\cdot \uni[8.31]{J.K^{-1}.mol^{-1}} 
\cdot \uni[300]{K}
}
{
\pi 
\cdot \uni[28 \times 10^{-3}]{kg.mol^{-1}}
\cdot \uni[32 \times 10^{-3}]{kg.mol^{-1}}
}
\Bigr\}
}
^{\frac{1}{2}}
\\ &\,&
\mbox{      \, \, \,       } 
\cdot
\frac{
21
}{
100
}
\cdot
\frac{
78
}{
100
}
{
\Bigr(
\frac{
\uni[1.01 \times 10^5]{J.m^{-3}}
}{
\uni[1.38 \times 10^{-23}]{J.K^{-1}}
\cdot \uni[8.31]{J.K^{-1}.mol^{-1}} 
}
\Bigr)
}^2
\\ &=&
\uni[3.438 \times 10^{37}]{s^{-1}.m^{-3}}
\\ &\approx&
\uni[3.4 \times 10^{37}]{s^{-1}.m^{-3}}
\label{}
\end{eqnarray*}
 %途中
\item
  気体分子と壁との衝突に関する以下の問に答えよ.

  \begin{enumerate}
  \def\labelenumii{\arabic{enumii}.}
  \item
    Maxwell-Boltzmann分布関数\(f(v_\mathrm{x})\)を用いて,1個の質量が\(m_\mathrm{a}\)の気体分子と面積\(A\)の壁との単位時間あたりの衝突数を表す式を導け.

    求める衝突数\(z\)は \begin{eqnarray*}
z 
&=& 
\mathcal{N}
\cdot A \cdot \overline{v_x}
\\ &=&
\frac{
N 
}{
V
}
\cdot
A
\int_{0}^{\infty}
v_x
f(v)
\,\diff x
\\ &=&
\frac{
N
}{
V
}
\cdot
A
%\cdot
\sqrt{
\frac{
m
}{
2 \pi KT
}
}
\cdot
\frac{1}{2}
\cdot
\frac{
2KT
}{
m
}
\\ &=&
\frac{
NA
}{
V
}
\sqrt{
\frac{
KT
}{
2 \pi m
}
}
\label{}
\end{eqnarray*}

  \item
    秤量した資料を半径\(r\)の円形小孔のある容器内で温度\(T(\si{\kelvin})\)に加熱したとき,時間\(\Delta t\)の間の重量減少は\(\Delta m\)であった.試料の蒸気圧\(p\)を求める式を導け.
  \end{enumerate}

  \begin{eqnarray*}
\Delta m =
\frac{
P
A_0
}{
(
2
\pi
m
K
T
)
}
\cdot
m
\cdot
\Delta t 
\label{}
\end{eqnarray*}
 より \begin{eqnarray*}
P =
\frac{
\Delta m
}{
r^2
\Delta t
}
\sqrt{
\frac{
2RT
}{
\pi
M
}
}
\label{}
\end{eqnarray*}


  \begin{enumerate}
  \def\labelenumii{\arabic{enumii}.}
  \setcounter{enumii}{2}
  \tightlist
  \item
    \SI{1000}{\degreeCelsius}において半径\SI{0.50}{mm}の小孔からの\ce{Ge}(モル質量\SI{72.6}{g.mol^{-1}})の重量損失が2時間で\uni[4.3 \times 10^{-2}]{mg}であった.\uni[1000]{\degreeCelsius}での\ce{Ge}の蒸気圧を求めよ.
  \end{enumerate}

  \begin{eqnarray*}
P &=&
\frac{
\uni[4.3 \times 10^{-2} \times 10^{-6}]{kg}
}{
(\uni[0.5 \times 10^{-3}]{m})^2
\cdot
\uni[7200]{s}
}
\sqrt{
\frac{
2
\cdot
\uni[8.31]{J.K^{-1}.mol^{-1}} 
\cdot 
\uni[1273]{K}
}{
\pi 
\cdot
\uni[72.6 \times 10^{-3}]{kg.mol^{-1}}
}
}
\\ &=&
\uni[7.27 \times 10^{-3}]{kg.m^{-1}.s^{-2}}
\\ &=&
\uni[7.27 \times 10^{-3}]{Pa}
= %\\ &=&
\uni[7.20 \times 10^{-8}]{atm}
\label{}
\end{eqnarray*}


  \begin{enumerate}
  \def\labelenumii{\arabic{enumii}.}
  \setcounter{enumii}{3}
  \tightlist
  \item
    1個の質量が\(m_\mathrm{a}\)の気体分子が小孔から流出する場合,初圧\(p_0\)の容器内の圧力変化を時間の関数として表せ.
  \end{enumerate}

  \begin{eqnarray*}
\frac{
\diff P
}{
\diff t
}
=
\frac{
\diff 
( 
\frac{
nRT
}{
V
}
)
}{
\diff t
}
=
\frac{
RT
}{
V
}
\frac{
\diff n
}{
\diff t 
}
=
\frac{
RT
}{
V
}
\cdot
\frac{
\diff 
%\Bigl( 
(
%\frac{N}{N_A}
N/N_\mathrm{A}
)
%\Bigr)
%\diff (N / N_A )
}{
\diff t
}
=
\frac{
RT
}{
N_\mathrm{A} V
}
\cdot
\frac{
\diff N
}{
\diff t
}
=
\frac{
KT
}{
V
}
\cdot
\frac{
\diff N
}{
\diff t
}
\label{}
\end{eqnarray*}
 %途中

  ここで

  %\begin{multicols}{2}
\begin{eqnarray*}
\frac{
  \diff N
  }{
  \diff t
  }
  =
  - Z_\mathrm{W}
  \cdot
  A_0
  =
  \frac{
    -P
    \cdot
    A_0      
  }{
    (
      2 \pi m K T
      )^{\frac{1}{2}}
  }
\label{}
\end{eqnarray*}
%\end{multicols}
 %途中

  よって

  %\begin{multicols}{2}
\begin{eqnarray*}
  \frac{
  \diff P
  }{
  \diff t
  }
  =
  \frac{
  KT
  }{
  V
  }
  \cdot
  \frac{
    -P
    \cdot
    A_0      
  }{
    (
      2 \pi m K T
      )^{\frac{1}{2}}
  }
\label{}
\end{eqnarray*}
%\end{multicols}
 %途中

  %\begin{multicols}{2}
\begin{eqnarray*}
  \int%_{0}^{\infty} 
      \frac{
      1
      }{
      P
      }
  \,\diff P
  &=&
  -
  \int%_{0}^{\infty} 
      {
      \Bigl( 
        \frac{KT}{2\pi m} 
      \Bigr) 
      }^\frac{
      1
      }{
      2
      }
      \cdot
      \frac{
      A_0
      }{
      V
      }
  \,\diff t
  \\ 
  \ln P 
  &=&
  -
  {
  \Bigl( 
    \frac{KT}{2\pi m} 
  \Bigr) 
  }^\frac{
  1
  }{
  2
  }
  \cdot
  \frac{
  A_0
  }{
  V
  }
  \cdot
  t
  +
  C
  \\
  P
  &=&
  \exp{C}
  \cdot
  \exp{
  \Bigl\{
  -
  {
  \Bigl( 
    \frac{KT}{2\pi m} 
  \Bigr) 
  }^\frac{
  1
  }{
  2
  }
  \cdot
  \frac{
  A_0
  }{
  V
  }  
  \cdot
  t
  \Bigl\}
  }
\label{}
\end{eqnarray*}
%\end{multicols}
 %途中

  ただし\(t = 0\)のとき\(P = P_0\),\(\exp{C} = P_0\)

  \begin{enumerate}
  \def\labelenumii{\arabic{enumii}.}
  \setcounter{enumii}{4}
  \tightlist
  \item
    内容積\uni[3.0]{m^3}の宇宙船に隕石が衝突して半径\uni[0.1]{mm}の穴が開いた.宇宙船の中の最初の酸素圧力が\uni[0.8]{atm}で,温度が\uni[298]{K}であったとすると,酸素圧力が\uni[0.7]{atm}まで下がるのにどれだけの時間がかかるか.
  \end{enumerate}

  %\begin{eqnarray*}

\label{}
\end{eqnarray*}

\end{enumerate}

\begin{enumerate}
\def\labelenumi{\arabic{enumi}.}
\setcounter{enumi}{6}
\tightlist
\item
  \ce{Ar}と\ce{Ne}の混合気体(圧力比で1:6)が小孔を通って流出するとき,最初にでてくる気体の混合比(圧力比)はいくらか,求めよ.
\end{enumerate}

\begin{eqnarray*}
\frac{Z_\mathrm{W} (\ce{Ar}) \times A_0}{Z_\mathrm{W} (\ce{Ne}) \times A_0}
=
\frac{P(\ce{Ar})}{P(\ce{Ne})}
\cdot
\frac
{1/\sqrt{m_{\ce{Ar}}}}
{1/\sqrt{m_{\ce{Ne}}}}
=
\frac{P(\ce{Ar})}{P(\ce{Ne})}
\cdot
\frac
{\sqrt{m_{\ce{Ne}}}}
{\sqrt{m_{\ce{Ar}}}}
=
\frac{1}{6}
\cdot
%\Bigl(
\frac{\sqrt{20.18}}{\sqrt{39.95}}
%\Bigr)
=
0.1184
\approx
\frac{
1
}{
8.44
}
\label{}
\end{eqnarray*}

\begin{eqnarray*}
\ce{Ar}:\ce{Ne} 
&=&
1:8.44
\\ &=&
1:8.4
\label{}
\end{eqnarray*}


\begin{enumerate}
\def\labelenumi{\arabic{enumi}.}
\setcounter{enumi}{7}
\item
  \SI{25}{\degreeCelsius}におけるアルゴン(分子量:39.95,衝突断面積:\SI{0.36}{nm^{2}})について,\SI{1.00}{atm}での拡散係数を計算せよ.また,ある管の中に\SI{0.10}{atm.cm^{-1}}の圧力勾配ができているとき,拡散によるアルゴンの流束はいくらか求めよ.ただし,アルゴンは完全気体として取り扱うものとする.
  \\

  拡散係数\(D\)は\begin{eqnarray*}
D
&=&
\frac{
1
}{
3
}
\lambda
\overline{c}
\\ &=&
\frac{
1
}{
3
}
\cdot
\frac{
KT
}{
\sqrt{
2
}
\rho
P
}
\cdot
\sqrt{
\frac{
8KT
}{
\pi
m
}
}
\\ &=&
\frac{
1
}{
3
}
\cdot
\sqrt{
\frac{
{
(
\uni[1.38 \times 10^{-23}]{J.K^{-1}}
\cdot
\uni[298]{K}
)
}^{
3
}
}{
2
\cdot
\pi
\cdot
\uni[39.95]{
g.mol^{-1}
}
}
}
\cdot
\frac{
1
}{
\uni[
0.36 \times
10^{-18}
]{
m^2
}
\cdot
\uni[1]{
atm
}
}
\\ &=&
\uni[1.541 \times 10^{-4}]{
J^{-\frac{3}{2}}.m^{-2}.atm^{-1}
}
\\ &=&
\uni[1.53 \times 10^{-19}]{
m.s^{-1}
}
\label{}
\end{eqnarray*}%途中

  流束\(J\)は%\begin{multicols}{2}
\begin{eqnarray*}
J
&=&
- D
\cdot
\frac
{
\diff \mathcal{N}
}{
\diff Z
}
\\
\mathcal{N}
&=&
\frac{
n
\cdot
N_\mathrm{A}
}{
V
}
\\ &=&
\frac{
n
\cdot
N_\mathrm{A}
}{
\frac{
nRT
}{
P
}
}
\\ &=&
\frac{
P
}{
KT
}
\\
\mbox{よって  } 
\frac{
\diff \mathcal{N}
}{
\diff Z
}
&=&
\frac{
1
}{
KT
}
\cdot
\frac{
\diff P
}{
\diff Z
}
\\ 
\mbox{ここで  } 
\frac{
\diff P
}{
\diff Z
}
&=&
\uni[0.1]{atm.cm^{-1}}
\mbox{より} 
\\
J
%&=&
=
- D
\cdot
\frac{
1
}{
KT
}
\cdot
\frac{
\diff P
}{
\diff Z
}
&=&
\uni[2.6 \times 10^{21}]{m^{-2}.s^{-1}}
\label{}
\end{eqnarray*}
%\end{multicols}

\end{enumerate}

%\printbibliography
%Testbibliography \cite{波多野恭弘2016}
\printbibliography[title=参考文献]

\end{document}
