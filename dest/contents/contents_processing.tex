\documentclass[uplatex,dvipdfmx,a4paper,10pt]{jsarticle}

\usepackage{geometry}
%\geometry{left=3cm, right=2cm, top=2cm, bottom=3cm}

%\usepackage[draft]{graphicx, color}
%https://qiita.com/zr_tex8r/items/442b75b452b11bee8049
%\usepackage\usepackage[draft]{graphicx}

% dvipdfmx は自動的に有効になる{graphicx}% dvipdfmx は自動的に有効になる
\usepackage[draft]{graphicx}

% dvipdfmx は自動的に有効になる% dvipdfmx は自動的に有効になる
\usepackage[dvipdfmx]{color}% dvipdfmx は自動的に有効になるらしい

\usepackage{multicol}
\usepackage{tikz}
\usepackage{float}
\usepackage[version=3]{mhchem}
\usepackage{chemfig}
\usepackage{bigfoot}
\usepackage{longtable} % 表組みに必要
\usepackage{booktabs} % 表組みに必要
\usepackage{subfig} % 図の横並び表示に必要
\usepackage{siunitx} % SI単位(国際単位系)を出力
\usepackage{here} 
\usepackage{pdfpages}

%http://konoyonohana.blog.fc2.com/blog-entry-58.html 
%エラー対応 “Too deeply nested.” 5階以上のネスト
\usepackage{enumerate}
\usepackage{enumitem}
\setlistdepth{20}
\renewlist{itemize}{itemize}{20}
\setlist[itemize]{label=\textbullet}



%https://ameblo.jp/h-krkr/entry-10908206471.html
\usepackage{mikibase}

\usepackage[%
	hidelinks,%
	pdfusetitle,%
	colorlinks=false,%
	bookmarks=true, % 以下ブックマークに関する設定
	bookmarksnumbered=true,%
	pdfborder={0 0 0},%
	bookmarkstype=toc,%
	pdftitle={「反応速度論」レポート課題 No.~2},%
	pdfauthor={otori334}%
]{hyperref}
\title{ 「反応速度論」レポート課題 No.~2}
\author{33417334 otori 334}
\usepackage{pxjahyper}
\usepackage{url}
\urlstyle{sf}


\usepackage[dvipdfmx]{pdfcomment}
\AtBeginDocument{\special{pdf:tounicode UTF8-UTF16}}
\usepackage{xcolor}
%\usepackage[x11names]{xcolor}
%\hypersetup{unicode}
\definecolor{myblue}{RGB}{187,254,237}

%\pdfmargincomment[icon=Note,color=myblue]{}


\usepackage{comment}

%\usepackage[backend=bibtex]{biblatex}
%\usepackage[backend=bibtex,style=numeric]{biblatex}
%\usepackage[backend=bibtex,style=ieee]{biblatex}
\usepackage[backend=biber,style=ieee]{biblatex}
%\bibliography{sample.bib}
\DeclarePrefChars{'-} % https://oku.edu.mie-u.ac.jp/tex/mod/forum/discuss.php?d=2313
\addbibresource{references.bib}



\newcommand*{\ppa}{%
	\ifthenelse{\boolean{mmode}}%
		{\mathrm{p}K_\mathrm{a}}%
		{\(\mathrm{p}K_\mathrm{a}\)}%
	}%

\newcommand*{\ppb}{%
	\ifthenelse{\boolean{mmode}}%
		{\mathrm{p}K_\mathrm{b}}%
		{\(\mathrm{p}K_\mathrm{b}\)}%
	}%

\newcommand*{\pph}{%
	\ifthenelse{\boolean{mmode}}%
		{\mathrm{pH}}%
		{\(\mathrm{pH}\)}%
	}%

\newcommand*{\diff}{%
	\ifthenelse{\boolean{mmode}}%
		{\mathrm{d}}%
		{\(\mathrm{d}\)}%
	}%

\usepackage{mathrsfs}%花文字
%筆記体はmathcal


%https://orumin.blogspot.com/2017/09/biblatex.html
\AtEveryBibitem{%
\ifentrytype{article}{%
}{}
\ifentrytype{inproceedings}{%
\clearfield{volume}
\clearfield{number}
}{}
\clearlist{publisher}
\clearfield{isbn}
\clearlist{location}
\clearfield{doi}
\clearfield{url}
}

\renewbibmacro{in:}{%
\ifentrytype{inproceedings}{%
  \setunit{}
  \addperiod\addspace In \textit{Proc.\ of the}}%
{\printtext{\bibstring{in}\intitlepunct}}
}

\renewbibmacro*{series+number:emphcond}{%
\ifentrytype{inproceedings}{%
  \printtext{(}\printfield{series}\printtext{)}\setunit*{\addcomma\space}\newunit}%
{%
  \printfield{number}%
  \setunit*{\addspace\bibstring{inseries}\addspace}%
  \ifboolexpr{%
    not test {\iffieldundef{volume}}
  }%
   {\printfield{series}}%
   {\ifboolexpr{%
       test {\iffieldundef{volume}}
       and
       test {\iffieldundef{part}}
       and
       test {\iffieldundef{number}}
       and
       test {\ifentrytype{book}}
     }%
      {\newunit\newblock}%
      {}%
    \printfield[noformat]{series}}%
  \newunit}
}




\usepackage[jis2004]{otf}


% prevent hyphenation
\hyphenpenalty=10000\relax
\exhyphenpenalty=10000\relax
\sloppy


%行間調整
%\renewcommand\baselinestretch{0.8} 

\usepackage{setspace} %% for spacing
%\begin{spacing}{0.7}
%hogehogehogehoge
%\end{spacing}

%脚注番号を (1)に変更
%\renewcommand\thefootnote{(\arabic{footnote})}
%脚注番号を 1)に変更
\renewcommand\thefootnote{\arabic{footnote})}
\newcommand{\vect}[1]{\mbox{\boldmath \(#1\)}}
\newcommand*{\uni}[2][]{
	\ifthenelse{\boolean{mmode}}{#1\,\si{#2}}{\(#1\,\si{#2}\)}}
\newcommand*{\pomment}[2][icon=Note,color=myblue]{
	\pdfmargincomment[#1]
	{#2}
	}
	
\renewcommand{\abstractname}{}
	
	
%\newcommand*{\myfig}[4][width=10cm]{\begin{figure}[H][H]\centering %\includegraphics[width=1.0\columnwidth][#1]{#2} \caption{#3} \label{fig:#4} \end{figure}}


% http://nos.hateblo.jp/entry/20081015/1224084491
%\newcommand{\fig}[3][width=15cm]{
%\begin{figure}[H][ht]
%\begin{center}
% \includegraphics[width=1.0\columnwidth][#1]{figs/#2}
%\end{center}
%\caption{#3}
%\label{fig:#2}
%\end{figure}}

%\newcommand{\fref}[1]{図\ref{fig:#1}}

%\renewcommand{\abstractname}{概要}

\def\tightlist{\itemsep1pt\parskip0pt\parsep0pt}
%\date{\date{提出日:\today}}
\date{提出日:\today}
%\setatomsep{1cm}
%\usepackage{luatexja-ruby}
% https://qiita.com/Selene-Misso/items/6c27a4a0947f10af3119o
%\subtitle{\vspace{10truept}{\large }}
%==========================================
% ドキュメント開始
%==========================================

\begin{document}

\maketitle

\begin{abstract}\end{abstract}
	








%\input{automatic_generated}
%\includepdf[nup=3x3,column,pages={1-43}]{01-06-2}
%\includepdf[nup=3x3,column,pages={1-68}]{01-09-2}

\begin{enumerate}
\def\labelenumi{\arabic{enumi}.}
\item
  \ce{nA -> P}なる気相反応において,Aの初圧を変えて初速度を測定したところ,初圧\uni[359]{torr}および\uni[152]{torr}のときに,初速度はそれぞれ\uni[1.50]{torr.s^{-1}},\uni[0.25]{torr.s^{-1}}であった.この反応の次数および速度定数を求めよ.
\item
  反応\ce{SO2Cl2 -> SO2 + Cl2 }は一次反応で,\uni[593]{\kelvin}において速度定数\uni[k = 2.2 \times 10^{-5}]{s^{-1}}である.\uni[593]{\kelvin}で2時間反応させると\ce{SO2Cl2}の何\si{\percent}が分解するか.
\item
  \ce{N2O5}の一次分解反応(\ce{N2O5 -> 2NO2 + 1/2 O2})の速度定数は,\uni[4.8 \times 10^{-4}]{s^{-1}}である.この反応の半減期はいくらか.また,最初\uni[0.5]{atm}あった圧力は,反応開始から10秒後にはいくらになるか.
\item
  放射性元素の崩壊速度は一次反応で表される.崩壊の半減期が1590年であるラジウムの崩壊定数(速度定数に相当)を求めよ.また,最初に1/4が崩壊するのに必要な年数を求めよ.
\item
  ある反応\ce{2A -> P}が,2次の速度式と速度定数\uni[k = 3.5 \times 10^{-4}]{M^{-1}.s^{-1}}をもつ.Aの濃度が\uni[0.260]{M}から\uni[0.011]{M}まで変化するのに要する時間を計算せよ.
\item
  \(t_{1/2}\)を半減期,\(t_{3/4}\)を反応基質濃度が初濃度の3/4まで減少する時間とおいた場合,\textit{n}次反応(\(n \geq 2\))での\(t_{1/2}/\,t_{3/4}\)の比を\textit{n}の関数として示せ.
\item
  \ce{CH3COOC2H5\mbox{(\textit{aq})} + OH^{-}\mbox{(\textit{aq})} -> CH3COO^{-}\mbox{(\textit{aq})} + C2H5OH\mbox{(\textit{aq})}}

  の反応の二次速度定数は,\uni[0.11]{M^{-1}.s^{-1}}である.酢酸エチルを水酸化ナトリウム水溶液に添加して初濃度が{[}\ce{NaOH}{]}=\uni[0.050]{M},{[}\ce{CH3COOC2H5}{]}=\uni[0.100]{M}になるようにした.反応を開始して10秒後の酢酸エチルの濃度はいくらか.
\item
  ある物質の分解の速度定数が\uni[30]{\degreeCelsius}で\uni[2.8 \times 10^{-3}]{M^{-1}.s^{-1}},\uni[50]{\degreeCelsius}で\uni[1.38 \times 10^{-2}]{M^{-1}.s^{-1}}であった.この反応のArrheniusパラメータ(頻度因子と活性化エネルギー)を求めよ.必ず単位をつけること.
\item
  \ce{NO}の気相酸化反応\ce{2NO + O2 ->[k_1] 2NO2}------\textcircled{\scriptsize 1}式は,下記の\textcircled{\scriptsize 2}式と\textcircled{\scriptsize 3}式の反応から成っている.\(k_1,k_2,k_{-2},k_3\)はそれぞれの反応の速度定数である.以下の問いに答えよ.
\end{enumerate}

\begin{multicols}{2}

\ce{2NO + O2 <=>[k_2][k_{-2}] N2O2}------\textcircled{\scriptsize 2}

\ce{N2O2 + O2 ->[k_3] 2NO2}------\textcircled{\scriptsize 3}

\end{multicols}

%\def\labelenumi{(\arabic{enumi})}

(1) \ce{N2O2}に定常状態近似を適用して\ce{NO2}生成の速度式を導け.

(2)
\ce{NO2}生成速度は,\ce{NO}分圧に2次,\ce{O2}分圧に1次であることが実験的に知られている.(1)で得た速度式がこの条件に適合する条件を示せ.
%\printbibliography
%Testbibliography \cite{波多野恭弘2016}
\printbibliography[title=参考文献]

\end{document}
