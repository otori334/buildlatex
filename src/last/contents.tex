\documentclass[uplatex,dvipdfmx,a4paper,10pt]{jsarticle}
\usepackage{geometry}
\geometry{%
	left=3cm,%
	right=2cm,%
	top=2cm,%
	bottom=3cm%
}
\usepackage{graphicx, color}
\usepackage{multicol}
\usepackage{tikz}
\usepackage{float}
\usepackage[version=3]{mhchem}
\usepackage{chemfig}
\usepackage{enumerate}
\usepackage{bigfoot}
\usepackage{longtable} % 表組みに必要
\usepackage{booktabs} % 表組みに必要
\usepackage{subfig} % 図の横並び表示に必要
\usepackage{siunitx} % SI単位(国際単位系)を出力
\usepackage{here} 

\usepackage[%
	hidelinks,%
	pdfusetitle,%
	colorlinks=false,%
	bookmarks=true, % 以下ブックマークに関する設定
	bookmarksnumbered=true,%
	pdfborder={0 0 0},%
	bookmarkstype=toc,%
	pdftitle={銅および亜鉛の電解析出とファラデー効率},%
	pdfauthor={}%
]{hyperref}
\title{}
\author{}
\usepackage{pxjahyper}
\usepackage{url}
\urlstyle{sf}


%\usepackage[backend=bibtex]{biblatex}
%\usepackage[backend=bibtex,style=numeric]{biblatex}
%\usepackage[backend=bibtex,style=ieee]{biblatex}
\usepackage[backend=biber,style=ieee]{biblatex}
%\bibliography{sample.bib}
\DeclarePrefChars{'-} % https://oku.edu.mie-u.ac.jp/tex/mod/forum/discuss.php?d=2313
\addbibresource{references.bib}



\newcommand*{\ppa}{%
	\ifthenelse{\boolean{mmode}}%
		{\mathrm{p}K_\mathrm{a}}%
		{\(\mathrm{p}K_\mathrm{a}\)}%
	}%

\newcommand*{\ppb}{%
	\ifthenelse{\boolean{mmode}}%
		{\mathrm{p}K_\mathrm{b}}%
		{\(\mathrm{p}K_\mathrm{b}\)}%
	}%

\newcommand*{\pph}{%
	\ifthenelse{\boolean{mmode}}%
		{\mathrm{pH}}%
		{\(\mathrm{pH}\)}%
	}%

\newcommand*{\diff}{%
	\ifthenelse{\boolean{mmode}}%
		{\mathrm{d}}%
		{\(\mathrm{d}\)}%
	}%




%https://orumin.blogspot.com/2017/09/biblatex.html
\AtEveryBibitem{%
\ifentrytype{article}{%
}{}
\ifentrytype{inproceedings}{%
\clearfield{volume}
\clearfield{number}
}{}
\clearlist{publisher}
\clearfield{isbn}
\clearlist{location}
\clearfield{doi}
\clearfield{url}
}

\renewbibmacro{in:}{%
\ifentrytype{inproceedings}{%
  \setunit{}
  \addperiod\addspace In \textit{Proc.\ of the}}%
{\printtext{\bibstring{in}\intitlepunct}}
}

\renewbibmacro*{series+number:emphcond}{%
\ifentrytype{inproceedings}{%
  \printtext{(}\printfield{series}\printtext{)}\setunit*{\addcomma\space}\newunit}%
{%
  \printfield{number}%
  \setunit*{\addspace\bibstring{inseries}\addspace}%
  \ifboolexpr{%
    not test {\iffieldundef{volume}}
  }%
   {\printfield{series}}%
   {\ifboolexpr{%
       test {\iffieldundef{volume}}
       and
       test {\iffieldundef{part}}
       and
       test {\iffieldundef{number}}
       and
       test {\ifentrytype{book}}
     }%
      {\newunit\newblock}%
      {}%
    \printfield[noformat]{series}}%
  \newunit}
}




\usepackage[jis2004]{otf}


% prevent hyphenation
\hyphenpenalty=10000\relax
\exhyphenpenalty=10000\relax
\sloppy


%行間調整
%\renewcommand\baselinestretch{0.8} 

\usepackage{setspace} %% for spacing
%\begin{spacing}{0.7}
%hogehogehogehoge
%\end{spacing}

%脚注番号を (1)に変更
%\renewcommand\thefootnote{(\arabic{footnote})}
%脚注番号を 1)に変更
\renewcommand\thefootnote{\arabic{footnote})}


\def\tightlist{\itemsep1pt\parskip0pt\parsep0pt}
%\date{}
%\setatomsep{1cm}
%\usepackage{luatexja-ruby}
% https://qiita.com/Selene-Misso/items/6c27a4a0947f10af3119o
%\subtitle{\vspace{10truept}{\large }}
%%%%%%%%%%%%%%%%%%%%%%%%%%%%%%%%%%%%%%%%%%%%%%%%%%%%%%%%%

\begin{document}

\maketitle

\begin{abstract}\end{abstract}
	
\tableofcontents
\newpage

%\input{automatic_generated}
\hypertarget{ux8981ux65e8}{%
\section{要旨}\label{ux8981ux65e8}}

本実験では.\cite{化学の新研究} 本実験では.\cite{1992分析化学}

\hypertarget{ux5b9fux9a13}{%
\section{実験}\label{ux5b9fux9a13}}

\hypertarget{ux7d50ux679cux3068ux89e3ux6790}{%
\section{結果と解析}\label{ux7d50ux679cux3068ux89e3ux6790}}

\hypertarget{ux8003ux5bdf}{%
\section{考察}\label{ux8003ux5bdf}}

\hypertarget{ux8a2dux554f}{%
\subsection{設問}\label{ux8a2dux554f}}

\begin{enumerate}
\def\labelenumi{\arabic{enumi}.}
\tightlist
\item
  \SI[per-mode=symbol]{0.5}{\mole \per \liter} -
  \ce{CuSO4}および\SI[per-mode=symbol]{0.5}{\mole \per \liter} -
  \ce{ZnSO4}溶液についての電流-電圧曲線および陰極分極曲線を示し,前者からはそれぞれの分解電圧を求めよ.
  ただし,電流の絶対値を対数プロットしたTafelプロット法を利用せよ.
\item
  \SI[per-mode=symbol]{0.5}{\mole \per \liter} -
  \ce{CuSO4}および\SI[per-mode=symbol]{0.5}{\mole \per \liter} -
  \ce{ZnSO4}溶液をある一定の電流密度で一定時間,定電流電解後,得られた電析物の重量を測定し,ファラデー効率を決定せよ.
\item
  \ce{CuSO4}溶液から銅を電解抽出する場合には\ce{CUSO4}の分解電圧以上の電圧が必要であるが,電解精製はこれよりずっと低い電圧で行うことができるのは何故か.
\item
  \ce{CuSO4}および\ce{ZnSO4}の混合溶液を白金電極で電解する時,陰極電位を次第に下げていくとどのような反応が起こると考えられるか.
\item
  \ce{Cu} -
  \ce{Zn}系平衡状態図を作製し,7-3黄銅および6-4黄銅について説明せよ.
\item
  空孔形成エネルギーが\SI{1}{\electronvolt}のとき,\ce{Cu}の融点直下における空孔濃度を求めよ.
\item
  格子定数aなる面心立方格子におけるa/2
  {[}110{]}なるバーガースベクトルの大きさを求めよ.
\item
  らせん転位1原子間隔の長さのもつひずみエネルギーがおよそ\SI{3}{\electronvolt}であることを示せ.\ce{Cu}の場合を考えて計算せよ.
  ただし,μ= 4.0 x 10\^{}11 dyn cm2,b = 2.5 x 108 cmとする.
\item
  強く冷間加工した金属中の転位密度は\({10^{12}}\)\,\si{\per \square \centi\metre}以上にも達する.
  この転位が一様に2原子距離(約\SI{5}{\angstrom})動いたときの塑性ひずみは何ほどになるか.
\end{enumerate}

%\printbibliography
%Testbibliography \cite{波多野恭弘2016}
\printbibliography[title=参考文献]

\end{document}
